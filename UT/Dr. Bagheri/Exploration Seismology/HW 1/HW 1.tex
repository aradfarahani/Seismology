\documentclass[letterpaper,11pt]{article}
\newlength{\outerbordwidth}
\raggedbottom
\raggedright
\usepackage[svgnames]{xcolor}
\usepackage{framed}
\usepackage{tocloft}
\usepackage{amsmath}
\usepackage{etoolbox}
\usepackage{fancyhdr}
\usepackage{geometry}
\usepackage{lastpage}
\usepackage{enumitem}
\usepackage{ragged2e}
\usepackage{hyperref}
\usepackage[normalem]{ulem}

\author{Mahdi Farmahini Farahani}


\hypersetup{hidelinks}

\setcounter{secnumdepth}{0}
\usepackage{titlesec}
\titlespacing{\subsection}{0pt}{*0}{*0}
\titlespacing{\subsubsection}{0pt}{*0}{*0}
\titleformat{\section}{\large\bfseries\uppercase}{}{}{}[\titlerule]
\titleformat*{\subsubsection}{\large\itshape}

\geometry{
	letterpaper,
	left=0.75in,
	right=0.75in,
	top=0.25in,
	bottom=0.5in,
	footskip=16pt
}

\pagestyle{fancy}
\fancyhf{}
\renewcommand{\headrulewidth}{0pt}
\renewcommand{\footrulewidth}{0.4pt}
\makeatletter
\renewcommand{\footrule}{%
	\vskip-3pt
	\hbox to\textwidth{%
		\hfil
		\color{footergray}\rule{\textwidth}{0.4pt}%
		\hfil
	}%
	\vskip2pt
}
\makeatother
\definecolor{footergray}{gray}{0.5}
\fancyfoot[L]{\color{footergray}\footnotesize Mahdi Farmahini Farahani (Arad)}
\fancyfoot[C]{\color{footergray}\footnotesize \thepage\ / \pageref{LastPage}}
\fancyfoot[R]{\color{footergray}\footnotesize \today}

% Custom commands
\robustify\cftdotfill
\setlength{\outerbordwidth}{3pt}
\definecolor{shadecolor}{gray}{0.75}
\definecolor{shadecolorB}{gray}{0.93}

\newcommand{\resitem}[1]{\item #1 \vspace{-2pt}}
\newcommand{\resheading}[1]{\vspace{8pt}
	\parbox{\textwidth}{\setlength{\FrameSep}{\outerbordwidth}
		\begin{shaded}
			\setlength{\fboxsep}{0pt}\framebox[\textwidth][l]{\setlength{\fboxsep}{4pt}\fcolorbox{shadecolorB}{shadecolorB}{\textbf{\sffamily{\mbox{~}\makebox[6.762in][l]{\large #1} \vphantom{p\^{E}}}}}}
		\end{shaded}
	}\vspace{-5pt}
}
\newcommand{\ressubheading}[4]{%
	\begin{tabular*}{\textwidth}{l@{\cftdotfill{\cftsecdotsep}\extracolsep{\fill}}r}
		\textbf{#1} & #2 \\
		\textit{#3} & \textit{#4} \\
	\end{tabular*}\vspace{-8pt}
}

%%%%%%%%%%%%%%%%%%%%%%%%%%%%%%%%%%%%%%%%%%%%%%%%%%%%%%%%%%%%%%%%%%%%%%%%%%%%%%%%%%%%%%%%%%%%%%%%%%%%%%%%%%

\begin{document}
	\thispagestyle{fancy}
	
	% Header block
	\begin{tabular*}{7in}{l@{\extracolsep{\fill}}r}
		\Large{Mahdi Farmahini Farahani} {\Large\textbf{(Arad)}} & \textbf{Student ID: 620104023} \\
		\href{mailto:aradfarahani@aol.com}{\texttt{E.mail: aradfarahani@aol.com}} & \href{https://aradfarahani.com/}{\texttt{Website: aradfarahani.com}} \\
		\texttt{Handle: @aradfarahani} & \href{https://orcid.org/0009-0008-3800-8688}{\texttt{ORCID: 0009-0008-3800-8688}}
	\end{tabular*}
	
	\vspace{1.9em}
	
	\vspace{-1.3em}
	\justifying
	\textbf{Part (a): Poisson’s ratio}
	\[
	\sigma_{ij} = \lambda \,\delta_{ij}\,\varepsilon_{kk} + 2\mu\,\varepsilon_{ij},
	\]
	where $\lambda$ and $\mu$ are the Lamé parameters, and $\varepsilon_{kk} = \varepsilon_{xx} + \varepsilon_{yy} + \varepsilon_{zz}$.
	
	Consider a uniaxial stress state,
	\[
	\sigma_{xx} = \sigma, \qquad \sigma_{yy} = \sigma_{zz} = 0.
	\]
	Due to symmetry of loading,
	\[
	\qquad \varepsilon_{yy} = \varepsilon_{zz} = \varepsilon_{t}.
	\]
	
	Apply Hooke's law to the $xx$-component:
	\[
	\sigma = \lambda(\varepsilon_{xx} + 2\varepsilon_{t}) + 2\mu\,\varepsilon_{xx}.
	\]
	
	Apply Hooke's law to the $yy$-component where stress is zero:
	\[
	0 = \lambda(\varepsilon_{xx} + 2\varepsilon_{t}) + 2\mu\,\varepsilon_{t}.
	\]
	
	Solve the second equation for $\varepsilon_{zz} = \varepsilon_{yy}$:
	\[
	\lambda\varepsilon_{xx} + 2(\lambda + \mu)\varepsilon_{yy} = 0,
	\]
	or
	\[
	\lambda\varepsilon_{xx} + 2(\lambda + \mu)\varepsilon_{zz} = 0,
	\]
	\[
	\varepsilon_{zz} = \varepsilon_{yy}= -\frac{\lambda}{2(\lambda + \mu)}\,\varepsilon_{xx}.
	\]
	
	Poisson's ratio ($\nu$) is defined as the negative ratio of lateral strain to longitudinal strain:
	\[
	\nu = -\frac{\varepsilon_{yy}}{\varepsilon_{xx}} = -\frac{\varepsilon_{zz}}{\varepsilon_{xx}}.
	\]
	
	Therefore,
	\[
	\boxed{\nu = \frac{\lambda}{2(\lambda + \mu)}}.
	\]
	\textbf{Part (b): Young’s modulus}


\[
0=\lambda(\varepsilon_{xx}+2\varepsilon_t)+2\mu\,\varepsilon_t
\quad\Rightarrow\quad
\lambda\varepsilon_{xx}+(2\lambda+2\mu)\varepsilon_t=0,
\]
so
\[
\varepsilon_t=-\frac{\lambda}{2(\lambda+\mu)}\,\varepsilon_{xx}.
\]

The \(xx\) component gives
\[
\sigma=\lambda(\varepsilon_{xx}+2\varepsilon_t)+2\mu\varepsilon_{xx}.
\]
Substitute \(\varepsilon_t\):
\[
\sigma
= \lambda\varepsilon_{xx} + 2\lambda\varepsilon_t + 2\mu\varepsilon_{xx}
= (\lambda+2\mu)\varepsilon_{xx}
+2\lambda\!\left(-\frac{\lambda}{2(\lambda+\mu)}\varepsilon_{xx}\right).
\]
Simplify:
\[
\sigma
= \Big[(\lambda+2\mu)-\frac{\lambda^2}{\lambda+\mu}\Big]\varepsilon_{xx}
= \frac{(\lambda+2\mu)(\lambda+\mu)-\lambda^2}{\lambda+\mu}\,\varepsilon_{xx}.
\]

\[
(\lambda+2\mu)(\lambda+\mu)-\lambda^2
= \lambda^2+3\lambda\mu+2\mu^2-\lambda^2
= \mu(3\lambda+2\mu).
\]
Therefore
\[
\sigma = \frac{\mu(3\lambda+2\mu)}{\lambda+\mu}\,\varepsilon_{xx}.
\]

By definition, Young's modulus is \(E=\dfrac{\sigma}{\varepsilon_{xx}}\) under uniaxial stress, so
\[
\boxed{ \;E=\frac{\mu(3\lambda+2\mu)}{\lambda+\mu}\; }.
\]

\end{document}
